\documentclass[journal, a4paper]{IEEEtran}

\usepackage{graphicx}   % Written by David Carlisle and Sebastian Rahtz
                        % Required if you want graphics, photos, etc.
                        % graphicx.sty is already installed on most LaTeX
                        % systems. The latest version and documentation can
                        % be obtained at:
                        % http://www.ctan.org/tex-archive/macros/latex/required/graphics/
                        % Another good source of documentation is "Using
                        % Imported Graphics in LaTeX2e" by Keith Reckdahl
                        % which can be found as esplatex.ps and epslatex.pdf
                        % at: http://www.ctan.org/tex-archive/info/
\usepackage{xcolor}
%\usepackage{psfrag}    % Written by Craig Barratt, Michael C. Grant,
                        % and David Carlisle
                        % This package allows you to substitute LaTeX
                        % commands for text in imported EPS graphic files.
                        % In this way, LaTeX symbols can be placed into
                        % graphics that have been generated by other
                        % applications. You must use latex->dvips->ps2pdf
                        % workflow (not direct pdf output from pdflatex) if
                        % you wish to use this capability because it works
                        % via some PostScript tricks. Alternatively, the
                        % graphics could be processed as separate files via
                        % psfrag and dvips, then converted to PDF for
                        % inclusion in the main file which uses pdflatex.
                        % Docs are in "The PSfrag System" by Michael C. Grant
                        % and David Carlisle. There is also some information
                        % about using psfrag in "Using Imported Graphics in
                        % LaTeX2e" by Keith Reckdahl which documents the
                        % graphicx package (see above). The psfrag package
                        % and documentation can be obtained at:
                        % http://www.ctan.org/tex-archive/macros/latex/contrib/supported/psfrag/

%\usepackage{subfigure} % Written by Steven Douglas Cochran
                        % This package makes it easy to put subfigures
                        % in your figures. i.e., "figure 1a and 1b"
                        % Docs are in "Using Imported Graphics in LaTeX2e"
                        % by Keith Reckdahl which also documents the graphicx
                        % package (see above). subfigure.sty is already
                        % installed on most LaTeX systems. The latest version
                        % and documentation can be obtained at:
                        % http://www.ctan.org/tex-archive/macros/latex/contrib/supported/subfigure/

\usepackage{url}        % Written by Donald Arseneau
                        % Provides better support for handling and breaking
                        % URLs. url.sty is already installed on most LaTeX
                        % systems. The latest version can be obtained at:
                        % http://www.ctan.org/tex-archive/macros/latex/contrib/other/misc/
                        % Read the url.sty source comments for usage information.

%\usepackage{stfloats}  % Written by Sigitas Tolusis
                        % Gives LaTeX2e the ability to do double column
                        % floats at the bottom of the page as well as the top.
                        % (e.g., "\begin{figure*}[!b]" is not normally
                        % possible in LaTeX2e). This is an invasive package
                        % which rewrites many portions of the LaTeX2e output
                        % routines. It may not work with other packages that
                        % modify the LaTeX2e output routine and/or with other
                        % versions of LaTeX. The latest version and
                        % documentation can be obtained at:
                        % http://www.ctan.org/tex-archive/macros/latex/contrib/supported/sttools/
                        % Documentation is contained in the stfloats.sty
                        % comments as well as in the presfull.pdf file.
                        % Do not use the stfloats baselinefloat ability as
                        % IEEE does not allow \baselineskip to stretch.
                        % Authors submitting work to the IEEE should note
                        % that IEEE rarely uses double column equations and
                        % that authors should try to avoid such use.
                        % Do not be tempted to use the cuted.sty or
                        % midfloat.sty package (by the same author) as IEEE
                        % does not format its papers in such ways.

\usepackage{amsmath}
\usepackage{listings}

\lstset{language=C} 



% Other popular packages for formatting tables and equations include:

%\usepackage{array}
% Frank Mittelbach's and David Carlisle's array.sty which improves the
% LaTeX2e array and tabular environments to provide better appearances and
% additional user controls. array.sty is already installed on most systems.
% The latest version and documentation can be obtained at:
% http://www.ctan.org/tex-archive/macros/latex/required/tools/

% V1.6 of IEEEtran contains the IEEEeqnarray family of commands that can
% be used to generate multiline equations as well as matrices, tables, etc.

% Also of notable interest:
% Scott Pakin's eqparbox package for creating (automatically sized) equal
% width boxes. Available:
% http://www.ctan.org/tex-archive/macros/latex/contrib/supported/eqparbox/

% *** Do not adjust lengths that control margins, column widths, etc. ***
% *** Do not use packages that alter fonts (such as pslatex).         ***
% There should be no need to do such things with IEEEtran.cls V1.6 and later.


% Your document starts here!
\begin{document}

% Define document title and author
	\title{Thread Pool Implementation in C}
	\author{Jialin Li and Edward Hu}
	\maketitle

% Write abstract here
\begin{abstract}
	This is a short report on our thread pool performance.
\end{abstract}

% Each section begins with a \section{title} command
\section{Introduction}
	% \PARstart{}{} creates a tall first letter for this first paragraph
	\PARstart{T}{hread} is a basic unit of CPU utilization, it consists of program counter, register set, and its own stack. Compared to processes, threads are much more lighweight since virtual memory space remains the same during a thread switch, which is not true for process switch. However, even threads sometimes pose overhead that will affect the performance of the program. Initializing and cleaning threads can pose a certain amount of overhead. For an parallel application that creates and destroys a large number of threads that each only runs for a short time, the thread management mechanism can create unnecessary resource waste. Thread pool can be useful in this type of scenario as it reduces the complexity of thread management and the overhead involved in thread creation and cleaning. The thread pool in implement in \textbf{GNU C}.

% Main Part
\section{Motivation}
	We want to be able to implement a thread pool that is convinient to use. We aim to create a provide an interface that is easy for programmers to use. Thus we minimize the number of functions exposed to the programmers to maintain simplicity. We also want the performance of the thread pool to be reasonably more efficient than normal thread-based version. Thus we create several test cases to compare the performances in different implementations. We mainly focused on testing the performance difference in thread-based implementation and Goroutine-based version for certain types of jobs. Finally, we simply want to learn how to implement a thread pool and explore the drawbacks along with possible improvmetns.\\

	There are several goals we are aiming to achieve in the experiment. First, we want to run both long and short tasks in the threal pool to analyze which type of tests have the most impact on the performance on the program. We've created created various tests for the thread pool to perform and we will show the data collected later in the analysis section. We hope the low overhead of switch between tasks will help us outperform thread-based implementation as well go-based implementation. In addition, we implemented work stealing mechanism to improve the work throughput done by the thread pool. We hope the work stealing function will further improve the efficiency of the thread pool. 
	
\section{Programming Interface}
The programming interface is designed for simplicity, there are three functions users are exposed to.\\

\begin{itemize}
	\item \begin{lstlisting}
void thread_pool_init(int workers, 
		      int mutex_flag);
	\end{lstlisting}
	This function initialize a new thread pool. The \textbf{workers} attribute specifies the number of threads that can exist in a thread pool. If $0$ is passd in, the number of threads will be initialized to the number of core in the execution environment. The \textbf{mutex\_flag} argument let user decides whether to use mutex or spinlock as synchronization method. Note if the thread pool is initialized more than once, every invocation of this function will be discarded.\\
	\item 
	\begin{lstlisting}
bool thread_pool_add(task_func *func,
		     void* aux,
		     enum CallType);
	\end{lstlisting}
	This function adds tasks to the thread pool. The \textbf{func} attribute represents the function that is to be executed. The \textbf{aux} attribute indicates the parameter \textbf{func} takes in. the \textbf{CallType} specifies whether to use blocking or non-blocking mechanism for the thread pool. If a task is added without failure, $true$ is returned. Otherwise, $false$ will be returned.\\
	\item \begin{lstlisting}
void thread_pool_wait();
	\end{lstlisting}
	This function signifies all threads in thread pool that no more job will be added and the calling function waits for all tasks in the thread pool to finish.
\end{itemize}

\section{Implementation}
The low overhead of the thread pool is acheived with several design decisions. We will discuss the major implementations. A thread pool is first initialized by the main thread and the corresponding number of threads are initialized with their attributes. Each task is represented as a \textbf{struct task}. It is created by the calling function whenever the user sends the function to be executed to the thread pool. It is then distributed to the threads in the thread pool in a round-robin style to ensure even distribution. Each thread owns a privat task queue to store all its assigned tasks. Each queue is protected by either a mutex or a spinlock depending on users' specification. However, it is possible that a thread's task queue is empty when it is first initialized. It will waste CPU resource if such task is kept up and running without executing any tasks. Thus, we use semaphore to improve the efficiency of the thread pool. Thus we let the main thread invoking sema\_post on each thread once it finishes adding tasks to the thread's task queue to indicate the availability of tasks. However, the thread's task queue can be empty for other reasons. The most important one is that the thread finishes executing all assigned tasks. Thus, we implement work stealing algorithm to keep the idle thread busy by letting it trying to steal tasks from other threads by randomly choosing one. Since we don't want tasks stealing to create too much overhead as it tries to grab tasks from other threads, we simply let the idle thread using pthread\_mutex\_trylock() or pthread\_spin\_trylock(). If the stealing succeeds, the tasks is grabbed from the victim's task queue and executed. Otherwise, the steal will fail. If the steal fails too many times, that indicates many threads' task queue is already empty and the thread pool is getting close to running out of tasks. In this case, letting more threads trying to steal tasks will create unnecessary overhead, thus the thread will be blocked to prevent wasting resources. If eventually the threads run out of tasks, it will increase \textbf{wait\_sema} to let the main thread know it finishes its job. If all threads signify that they've finihsed their tasks, then thread pool is terminated.

\section{Analysis}

	% This is how you define a table: the [!hbt] means that LaTeX is forced (by the !) to place the table exactly here (by h), or if that doesnt work because of a pagebreak or so, it tries to place the table to the bottom of the page (by b) or the top (by t).
	%\begin{table}[!hbt]
		% Center the table
	%\begin{center}
	%	\caption{Simulation Parameters}
	%	\label{tab:simParameters}
		% Table itself: here we have two columns which are centered and have lines to the left, right and in the middle: |c|c|
	%	\begin{tabular}{|c|c|}
	%		% To create a horizontal line, type \hline
	%		\hline
	%		% To end a column type &
	%		% For a linebreak type \\
	%		Information message length & $k=16000$ bit \\
	%		\hline
	%		Radio segment size & $b=160$ bit \\
	%		\hline
	%		Rate of component codes & $R_{cc}=1/3$\\
	%		\hline
	%		Polynomial of component encoders & $[1 , 33/37 , 25/37]_8$\\
	%		\hline
	%	\end{tabular}
	%	\end{center}
	%\end{table}


	% If you have questions about how to write mathematical formulas in LaTeX, please read a LaTeX book or the 'Not So Short Introduction to LaTeX': tobi.oetiker.ch/lshort/lshort.pdf

	% This is how you include a eps figure in your document. LaTeX only accepts EPS or TIFF files.
	\begin{figure}[!hbt]
		% Center the figure.
		\begin{center}
		% Include the eps file, scale it such that it's width equals the column width. You can also put width=8cm for example...
		\includegraphics[width=\columnwidth]{plot_tf}
		% Create a subtitle for the figure.
		\caption{Simulation results on the AWGN channel. Average throughput $k/n$ vs $E_s/N_0$.}
		% Define the label of the figure. It's good to use 'fig:title', so you know that the label belongs to a figure.
		\label{fig:tf_plot}
		\end{center}
	\end{figure}

\section{Filling this page}
	Gallia est omnis divisa in partes tres, quarum unam incolunt Belgae, aliam Aquitani, tertiam qui ipsorum lingua Celtae, nostra Galli appellantur. Gallos ab Aquitanis Garumna flumen, a Belgis Matrona et Sequana dividit. Horum omnium fortissimi sunt Belgae, propterea quod a cultu atque humanitate provinciae longissime absunt, minimeque ad eos mercatores saepe commeant atque ea quae ad effeminandos animos pertinent important, proximique sunt Germanis, qui trans Rhenum incolunt, quibuscum continenter bellum gerunt. Qua de causa Helvetii quoque reliquos Gallos virtute praecedunt, quod fere cotidianis proeliis cum Germanis contendunt, cum aut suis finibus eos prohibent aut ipsi in eorum finibus bellum gerunt. Eorum una, pars, quam Gallos obtinere dictum est, initium capit a flumine Rhodano, continetur Garumna flumine, Oceano, finibus Belgarum, attingit etiam ab Sequanis et Helvetiis flumen Rhenum, vergit ad septentriones. Belgae ab extremis Galliae finibus oriuntur, pertinent ad inferiorem partem fluminis Rheni, spectant in septentrionem et orientem solem.

\section{Conclusion}
	This section summarizes the paper.

% Now we need a bibliography:
\begin{thebibliography}{5}

	%Each item starts with a \bibitem{reference} command and the details thereafter.
	\bibitem{HOP96} % Transaction paper
	J.~Hagenauer, E.~Offer, and L.~Papke. Iterative decoding of binary block
	and convolutional codes. {\em IEEE Trans. Inform. Theory},
	vol.~42, no.~2, pp.~429–-445, Mar. 1996.

	\bibitem{MJH06} % Conference paper
	T.~Mayer, H.~Jenkac, and J.~Hagenauer. Turbo base-station cooperation for intercell interference cancellation. {\em IEEE Int. Conf. Commun. (ICC)}, Istanbul, Turkey, pp.~356--361, June 2006.

	\bibitem{Proakis} % Book
	J.~G.~Proakis. {\em Digital Communications}. McGraw-Hill Book Co.,
	New York, USA, 3rd edition, 1995.

	\bibitem{talk} % Web document
	F.~R.~Kschischang. Giving a talk: Guidelines for the Preparation and Presentation of Technical Seminars.
	\url{http://www.comm.toronto.edu/frank/guide/guide.pdf}.

	\bibitem{5}
	IEEE Transactions \LaTeX and Microsoft Word Style Files.
	\url{http://www.ieee.org/web/publications/authors/transjnl/index.html}

\end{thebibliography}

% Your document ends here!
\end{document}